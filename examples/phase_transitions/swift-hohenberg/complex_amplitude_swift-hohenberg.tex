\documentclass[11pt]{article}
\usepackage{amsmath,amssymb,siunitx}
\usepackage[letterpaper,top=1.25in,bottom=1.25in,left=1in,right=1in]{geometry}

\title{Complex Amplitude Model for Swift-Hohenberg Equation}
\author{Trevor Keller and Nana Ofori-Opoku}
\date{\today}

\begin{document}
\maketitle

This example applies the Newell-Whitehead-Segel amplitude equation,
which can be considered as a complex generalization of the Swift-Hohenberg equation.
The model includes three complex amplitude fields $\{A\}$ with three lattice vectors $\{\vec{k}\}$.

The system represented in MMSP using a grid of vectors of complex scalars.
The initial condition sets a seed of radius $r=10\Delta x$ at the center,
with initial values of $(\num{0.35 + 0i}, \num{0.35 + 0i}, \num{0.35 + 0i})$
inside the seed and $(\num{0 + 0i}, \num{0 + 0i}, \num{0 + 0i})$ outside.
The mesh resolution $\Delta x=\Delta y=\frac{\pi}{2}$.

The equations of motion are then
\begin{align}
	\frac{\partial A_i}{\partial t} &= -\left[- \epsilon A_i
	                                         + \gamma |A_i|^2 A_i
	                                         + \left(\nabla^2 + \num{2i}\hat{k}_i\cdot\vec{\nabla}\right)^2 A_i\right]\\
	                                &= \left(\epsilon - \gamma |A_i|^2\right) A_i
	                                 - \left(\nabla^2 + \num{2i}\hat{k}_i\cdot\vec{\nabla}\right)^2 A_i.	                                         
\end{align}
The equation is solved using $\gamma = 1$, $\epsilon = 0.5$, $\vec{k}_1 = \left(\frac{ \sqrt{3}}{2}, -\frac{1}{2}\right)$,
                                                             $\vec{k}_2 = \left(0, 1\right)$,
                                                             $\vec{k}_3 = \left(\frac{-\sqrt{3}}{2}, -\frac{1}{2}\right)$,
                                                             and $\Delta t = \num{e-4}$.
$|A_i|$ denotes the norm or complex modulus of $A_i$.
The operator $\left(\nabla^2 + \num{2i}\hat{k}_i\cdot\vec{\nabla}\right)^2$ is split into
two applications of $\left(\nabla^2 + \num{2i}\hat{k}_i\cdot\vec{\nabla}\right)$ using an intermediate grid.
\end{document}
