\documentclass[11pt]{article}
\usepackage{amsmath,amssymb,siunitx}
\usepackage[letterpaper,top=1in,bottom=1in,left=0.75in,right=0.75in]{geometry}

\title{Complex Amplitude Model for Swift-Hohenberg Equation}
\author{Trevor Keller (borrowing heavily from Nana Ofori-Opoku)}
\date{\today}

\begin{document}
\maketitle

This example applies the Newell-Whitehead-Segel amplitude equation,
which can be considered as a complex generalization of the Swift-Hohenberg equation.
The model includes three complex amplitude fields $\{A\}$ with three lattice vectors $\{\vec{k}\}$.

The system represented in MMSP using a grid of vectors of complex scalars.
The initial condition sets a seed of radius $r=10\Delta x$ at the center,
with initial values of $(\num{0.35 + 0i}, \num{0.35 + 0i}, \num{0.35 + 0i})$
inside the seed and $(\num{0 + 0i}, \num{0 + 0i}, \num{0 + 0i})$ outside.

\section{Free Energy}
\[
	\mathcal{F} = \int_\Omega\left[  \right]\mathrm{d}V
\]


\section{Equations of Motion}
The equations of motion are
\begin{align}
	\frac{\partial A_i}{\partial t} &= -M_i\left[\left(\epsilon - 2\beta\psi + 3\psi^2\right)A_i
	                                   + \left(2\beta - 6\psi\right)\prod\limits_{j\neq i}^{3}A_j^*
	                                   + 3\left(|A_i|^2 + 2\sum\limits_{j\neq i}^3 |A_j|^2\right) A_i
	                                   + \left(\nabla^2 + \num{2i}\vec{k}_i\cdot\vec{\nabla}\right)^2A_i\right]
\end{align}
$|A_i|$ denotes the norm or complex modulus of $A_i$.
The operator $\left(\nabla^2 + \num{2i}\vec{k}_i\cdot\vec{\nabla}\right)^2$ is performed by nested
applications of the covariant gradient operator $\left(\nabla^2 + \num{2i}\vec{k}_i\cdot\vec{\nabla}\right)$ using an intermediate grid.

\begin{table}[!ht]\centering
	\begin{tabular}{lll}\hline
		Parameter        & Symbol       & Value\\\hline
		``temperature''  & $\epsilon$   & $-0.4$\\
		                 & $\beta$      & $-0.1$\\
		Average density  & $\psi$     & $-0.2$\\
		Mobility         & $M_i$        & $1.0$\\
		Lattice vector 1 & $\vec{k}_1$  & $\left(\frac{\sqrt{3}}{2}, -\frac{1}{2}\right)$\\
		Lattice vector 2 & $\vec{k}_2$  & $\left(0, -1\right)$\\
		Lattice vector 3 & $\vec{k}_3$  & $\left(-\frac{\sqrt{3}}{2}, -\frac{1}{2}\right)$\\
		Mesh resolution  & $h=\Delta x$ & $\frac{\pi}{2}$\\\hline
	\end{tabular}
\end{table}

\section{Density}
To convert from amplitudes back to density -- to see the lattice -- apply the mapping
\[
	\Psi = \psi + 2\sum_{j=1}^3\left[\Re(A_j)\cos\left(\vec{k}_j\cdot\vec{r}\right) - \Im(A_j)\sin\left(\vec{k}_j\cdot\vec{r}\right)\right],
\]
where $\Re$ is the real part and $\Im$ the imaginary part of the complex scalar at each mesh point.

\end{document}
